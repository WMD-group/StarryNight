%JACS TEMPLATE STARTS HERE
\documentclass[journal=jacsat,manuscript=communication]{achemso}
\usepackage[version=3]{mhchem} % Formula subscripts using \ce{}
\usepackage{lineno,xcolor}
\newcommand*{\mycommand}[1]{\texttt{\emph{#1}}}

%\documentclass[aip,graphicx]{revtex4-1}
%\documentclass[aip,apl,amsmath,amssymb,linenumbers,reprint]{revtex4-1}
%\documentclass[aip,apl,amsmath,amssymb,linenumbers,preprint]{revtex4-1}
\usepackage{graphicx}
\usepackage[version=3]{mhchem} % Formula subscripts using \ce{}
\usepackage{subfig}
\usepackage{dcolumn}% Align table columns on decimal point

\usepackage{amsmath}% amsmath...
\usepackage{bm}% bold math

\usepackage{siunitx}% JMF addition - once a physicist, always a physicist

%\bibliographystyle{aipnum4-1}
%\draft % marks overfull lines with a black rule on the right

\title{Starry Night: Or how I learned to stop worrying and love perovskites}

\author{Jarvist M. Frost}
\author{Keith T. Butler}
%\author{Federico Brivio}
%\author{Christopher H. Hendon}
\affiliation{Centre for Sustainable Chemical Technologies and Department of Chemistry, University of Bath, Claverton Down, Bath BA2 7AY, UK}

%\author{Mark van Schilfgaarde}
%\affiliation{Department of Physics, Kings College London, London WC2R 2LS, UK}

\author{Aron Walsh}
\email{a.walsh@bath.ac.uk}
\affiliation{Centre for Sustainable Chemical Technologies and Department of Chemistry, University of Bath, Claverton Down, Bath BA2 7AY, UK}

\begin{document}

\begin{abstract}
Abstract. EPSRC gave us some money so we did our best to do great science, and these are our conclusions. 
\end{abstract}

%\pacs{88.40.-j, 71.20.Nr, 72.40.+w, 61.66.Fn}
% 71.20.Nr 	Semiconductor compounds 
% 72.40.+w 	Photoconduction and photovoltaic effects
% 61.66.Fn 	Inorganic compounds 
% 88.40.jn 	Thin film Cu-based I-III-VI2 solar cells
% 88.40.-j 	Solar energy

%\maketitle 

Introduction

Solar cells from hybrid perovskite material display interesting device physics, most notably a hysterisis loop in the JV curve.

In this paper we will
\begin{itemize}
 \item suggest how the unique behaviour of hybrid perovskite solar cells arrises from ferroelectric domains
 \item develop a monte-carlo code to model the ferroelectric behaviour of hybrid perovskites from a model classical Hamiltonian
 \item use this code to simulate device physics, hysterisis and built-in potential as a function of temperature
\end{itemize}

Method

Our Hamiltonian looks a little like this

\begin{align*}
H = &\sum^n_{dipole,E-field} &\frac{1}{4.\pi \epsilon_0} (p_i.E) \\
+ &\sum^{n,m}_{dipole,dipole} &\frac{1}{4.\pi \epsilon_0} (\frac{p_i.p_j}{r^3}-\frac{3(\hat{n}.p_i)(\hat{n}.p_j)}{r^3}) \\
+ &\sum^n_{dipole,strain} &K.|p_i.\hat{x}|
\end{align*}  

Results

Conclusion

%\begin{figure*}[ht!]
%\begin{center}
%\resizebox{15 cm}{!}{\includegraphics*{dipoles.pdf}}
%\caption{\label{fig-dip} Schematic perovskite crystal structure of MAPbI$_3$ (\textbf{a}), and the possible orientations of molecular dipoles within the lattice (\textbf{b}). Note that MA has an associated molecular dipole of 2.3 Debye, a fundamental difference compared to the spherical cation symmetry in inorganic perovskites such as CsSnI$_3$.} 
%\end{center}
%\end{figure*}

\begin{acknowledgement}
We acknowledge membership of the UK's HPC Materials Chemistry Consortium, which is funded by EPSRC grant EP/F067496. 
J.M.F. and K.T.B. are funded by EPSRC Grants EP/K016288/1 and EP/J017361/1, respectively.
%F.B. is funded through the EU DESTINY Network (Grant 316494).
%C.H.H. is funded by ERC (Grant 277757). 
A.W. acknowledges support from the Royal Society and ERC (Grant 277757). 
We are grateful for the lyrical encouragment of Salt N Pepa. 
\end{acknowledgement}

%\begin{suppinfo}
%    GAUSSIAN input and output files used in this work are electronically available in a Figshare dataset.%\cite{figshare}
%any notes on the suppinfo...
%create a file store on figshare + get a DOI?
%\end{suppinfo}

\bibliography{library}

\end{document}
